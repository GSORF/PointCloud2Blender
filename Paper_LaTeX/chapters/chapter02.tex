\section{Concept}

Lorem ipsum dolor sit amet, consectetur adipiscing elit. Nullam hendrerit interdum sagittis. Nulla facilisi. Pellentesque laoreet tincidunt semper. Pellentesque pellentesque lectus id arcu interdum cursus ut ac dui. Etiam feugiat nisl ac odio suscipit pretium venenatis eget diam. Morbi molestie ipsum sit amet sapien rutrum luctus. Proin ac dolor ut metus laoreet aliquam non id nunc. Curabitur non efficitur dolor. Donec iaculis, dui et porta iaculis, magna tellus placerat ex, sed porta sem ligula ac augue. Fusce vitae sagittis ex. Pellentesque faucibus cursus elit, et faucibus velit cursus in. Nam lobortis id neque id hendrerit. Fusce et dolor nisi.


\subsection{Use case diagram}

\begin{figure}[h]
	\centering
	\includegraphics[scale=0.4]{UseCaseDiagram_PC2B.png}
	\caption{Use Case Diagram}
	\label{fig:use_case}
\end{figure}


\subsection{Laser scanning on location}

Cum sociis natoque penatibus et magnis dis parturient montes, nascetur ridiculus mus. Donec non auctor sem, sit amet fringilla purus. Phasellus eu orci et nibh lobortis faucibus id vel lorem. Aliquam ut diam id mi aliquam finibus eu id neque. Nam consequat efficitur mi sed maximus. Nullam egestas neque enim. Nulla nec eleifend mauris, eget sollicitudin velit. Quisque ultricies feugiat neque ut condimentum. Aliquam vehicula faucibus sapien non convallis. Nullam consectetur sagittis sollicitudin. Nulla mollis laoreet metus et consectetur.

\begin{figure}[h]
	\centering
	\includegraphics[scale=0.4]{PellerhausLaserScan.jpg}
	\caption{Scanning with Faro Focus 3D}
	\label{fig:laser_scanning_on_location}
\end{figure}



\begin{table}[h]
	\centering
	\begin{tabular}{l | l | l}
		A & B & C \\
		\hline
		1 & 3 & 4 \\
		1 & 3 & 4 \\
	\end{tabular}
	\caption{very basic table caption}
	\label{tab:abc}
\end{table}



\section{Generating data and testing algorithms}

\subsection{BlenSor}

Etiam non volutpat diam. Nam ac consectetur felis. Ut nec mi dictum, lobortis mauris quis, dapibus ligula. Nulla porttitor diam sed mauris dapibus posuere. Fusce pellentesque odio at nisl placerat porta. Donec urna risus, iaculis vitae justo quis, tempus ullamcorper diam. Integer eu gravida est. Phasellus eu ex tincidunt urna tempus pulvinar in in metus. Mauris tempus magna ac finibus suscipit. Praesent malesuada magna nibh, at rutrum felis semper a.

\subsection{Test-Addon for Blender}

Etiam non volutpat diam. Nam ac consectetur felis. Ut nec mi dictum, lobortis mauris quis, dapibus ligula. Nulla porttitor diam sed mauris dapibus posuere. Fusce pellentesque odio at nisl placerat porta. Donec urna risus, iaculis vitae justo quis, tempus ullamcorper diam. Integer eu gravida est. Phasellus eu ex tincidunt urna tempus pulvinar in in metus. Mauris tempus magna ac finibus suscipit. Praesent malesuada magna nibh, at rutrum felis semper a.

\section{Prototype}

\subsection{Point Cloud Importer}

Etiam non volutpat diam. Nam ac consectetur felis. Ut nec mi dictum, lobortis mauris quis, dapibus ligula. Nulla porttitor diam sed mauris dapibus posuere. Fusce pellentesque odio at nisl placerat porta. Donec urna risus, iaculis vitae justo quis, tempus ullamcorper diam. Integer eu gravida est. Phasellus eu ex tincidunt urna tempus pulvinar in in metus. Mauris tempus magna ac finibus suscipit. Praesent malesuada magna nibh, at rutrum felis semper a.

\subsubsection{Point Cloud data formats}

\begin{table}[h]
	\centering
	\begin{subtable}[h]{0.45\textwidth}
		\centering
		\begin{tabular}{l | l | l}
			Day & Max Temp & Min temp \\
			\hline \hline
			Mon & 20 & 13 \\
			Tue & 22 & 14 \\
			Wed & 23 & 12 \\
			Thu & 25 & 13 \\
			Fri & 18 & 7 \\
			Sat & 15 & 13 \\
			Sun & 20 & 13
		\end{tabular}
		\caption{First Week}
		\label{tab:week1}
	\end{subtable}
	\hfill
	\begin{subtable}[h]{0.45\textwidth}
		\centering
		\begin{tabular}{l | l | l}
			Day & Max Temp & Min Temp \\
			\hline \hline
			Mon & 17 & 11 \\
			Tue & 16 & 10 \\
			Wed & 14 & 8 \\
			Thu & 12 & 5 \\
			Fri & 15 & 7 \\
			Sat & 16 & 12 \\
			Sun & 15 & 9
		\end{tabular}
		\caption{Second Week}
		\label{tab:week2}
	\end{subtable}
	\caption{Max and min temp recorded during the first two weeks in January}
	\label{tab:temps}
\end{table}


\subsection{Projecting 3D points onto a 2D plane}


\subsection{Saving textures}


\subsection{OpenGL Point Cloud Viewer}

This russian video tutorial was very helpful with the basic setup with the Qt framework.

\cite{ytQtOpenGL}

\subsection{Meshing}

Using the current pixel inside two for-loops in combination with the neighboring pixels to the right, bottom-right and bottom makes up a quad, which can be textured.

\subsection{Texture Coordinates and Normals}

Texture Coordinates go from 0.0 to 1.0 in the x and y direction, respectively. Usually the texture coordinate axes are referred to as s and t. By dividing the current coordinate by the width and the height of the image, respectively, the coordinates can be normalized.

Calculating normals is accomplished by using the cross product of the two vectors forming the current quad.

\subsection{Mesh Exporter}

There are different formats, one had to be chosen that supported at least vertices and faces.

\subsubsection{.obj}

The .obj format is the most popular and can be one of the easiest to understand file formats to save 3D geometry with not only points, but vertices, normals, texture coordinates and much more. It was the first choice when testing mesh exporting from the converter software and examining it in Blender.

\subsubsection{.blend}

A personal goal for this research was to implement a .blend export feature to allow for a native importing of the panorama mesh into Blender. However, this goal was not reached in this project. As it turned out, exporting the binary Blender file format was quite complicated, due to it's versatile structure. An experienced Blender Developer, Jeroen Bakker, stated in 2009 “When implementing loading and saving blend-files in a custom tool the difficulty is the opposite. In a custom tool loading a blend-file is easy, and saving a blend-file is difficult.” \parencite[see]{webMysteryOfTheBlend}. At least implementing it with the limited time for the thesis it was not feasable.



\subsubsection{custom format}

Even the Blender community suggested to not use the .blend format directly, but rather try a custom binary format. \parencite[compare]{webBlenderArtistsBlendExport}


\subsection{Optimizations}

The initial algorithms and approaches had some flaws, which needed to get eliminated to get a clean mesh out of the converter. Those are presented as follows:

\subsubsection{Flip horizontal direction of panorama}

The panorama is flipped horizontally.

\subsubsection{Panorama pixel depth testing}

It can happen that two points from the point cloud happen to result in the same pixel in the 2D panorama. This might result in a noisy image result, if not handled with care. To avoid any errors, it is important to take only the closest point to the camera, instead of letting every point override the corresponding pixel in the image.

\subsubsection{Panorama noise reduction}

Since there is only a limited number of points, the panorama texture gets quite noisy, especially with a higher resolution option set in the converter. A harsh change from light gray to black values in the depth map will result in a noisy 3D structure as well.
To solve this issue, the image is blurred by a user setting or automatically (TODO!).

\subsubsection{Remove doubles}

The meshing technique resulted in a very high point count for the .obj file. Example: For a 4x resolution panorama with 2,198,528 vertices, using the "remove doubles" option in Blender 3D automatically removed 2,100,716 vertices.
Solution: several passes for vertices, texture coordinates and normals (TODO!).

\subsubsection{3D Distortion}

The generated 3D mesh from the 2D panorama results in a distorted one, the more it touches the top.
Solution: None yet.

\subsubsection{Tiling}

Due to the higher resolution meshes having several megabytes in size and taking some time to import in Blender, this has to be optimized somehow.
Solution: Create tiles when higher resolution is set. E.g. with a 4x resolution, create four tiles (that's four seperate .obj files).