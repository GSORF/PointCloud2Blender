\section{Conclusion}

To conclude, in this work we presented a method to process a point cloud file with a custom software to get a textured mesh ready to use in Blender (or any other 3d software package) and examined the use of the generated mesh for designers with the use case of reconstrucing the historic Pellerhaus Nürnberg.
This approach led to an easy way to get a mesh from a point cloud. The goals have been met and the reaults can be worked with (TODO: correct grammar and spelling ;) ).

\subsection{Mesh generation}

Lorem ipsum dolor sit amet, consectetur adipiscing elit. Nullam hendrerit interdum sagittis. Nulla facilisi. Pellentesque laoreet tincidunt semper. Pellentesque pellentesque lectus id arcu interdum cursus ut ac dui. Etiam feugiat nisl ac odio suscipit pretium venenatis eget diam. Morbi molestie ipsum sit amet sapien rutrum luctus. Proin ac dolor ut metus laoreet aliquam non id nunc. Curabitur non efficitur dolor. Donec iaculis, dui et porta iaculis, magna tellus placerat ex, sed porta sem ligula ac augue. Fusce vitae sagittis ex. Pellentesque faucibus cursus elit, et faucibus velit cursus in. Nam lobortis id neque id hendrerit. Fusce et dolor nisi.

\subsection{Handling non-LiDAR-Data}

Although the goal for this work was to 

Additionally, Mrs. Eichhorn, a fellow student finishing her bachelor's degree in October 2015, teamed up with me to examine ways how this research can be used with a depth sensor. She wrote a custom tool to save the output of a depth sensor to an .xyz file which is compatible with the PC2B converter software. The results are great, showing that it is possible to generate 3D panoramas from depth sensors as well. The test data has XXX points and is processed in YYY milliseconds in the PC2B converter. [...]


-Title: Selbstständige Navigation eines autonomen Roboters mittels 3D-Tiefensensor
-Description: .... kommt noch ....
-Deadline:  13.August will ich abgeben, aber spätestens geht auch 16.10
-Colored Point Cloud file: ... wird auch nachgereicht ...




\section{Future Work}

There are some further ideas how the software and how the 3D model can be used in other fields of application.

\subsection{Realtime Conversion}

Etiam non volutpat diam. Nam ac consectetur felis. Ut nec mi dictum, lobortis mauris quis, dapibus ligula. Nulla porttitor diam sed mauris dapibus posuere. Fusce pellentesque odio at nisl placerat porta. Donec urna risus, iaculis vitae justo quis, tempus ullamcorper diam. Integer eu gravida est. Phasellus eu ex tincidunt urna tempus pulvinar in in metus. Mauris tempus magna ac finibus suscipit. Praesent malesuada magna nibh, at rutrum felis semper a.

\subsection{Integration in the Blender core}

Aliquam at varius elit. Suspendisse viverra ex a ipsum scelerisque condimentum. Nam ac neque luctus, ullamcorper neque sit amet, tincidunt felis. Maecenas id orci rutrum, tincidunt nisl non, aliquam sapien. Aenean vestibulum erat ut nisl vehicula, quis rutrum eros fermentum. Maecenas ut arcu elit. Cras vestibulum pharetra facilisis. Quisque non elit iaculis leo dignissim faucibus. Etiam fringilla tortor nibh, sit amet volutpat diam consequat ac. Sed consectetur metus quis suscipit sollicitudin. Morbi imperdiet scelerisque nunc, in dapibus lacus. Praesent ullamcorper condimentum augue vitae finibus. 

\subsection{3D Lenticular}

Interdum et malesuada fames ac ante ipsum primis in faucibus. Cras quis pharetra libero. Pellentesque consectetur, quam vel ultrices finibus, sem enim consectetur mi, in dictum tellus leo eu ante. Maecenas consequat egestas erat, in vestibulum velit pulvinar ac. Suspendisse ullamcorper augue sapien, ac suscipit nulla dictum in. Nam sit amet congue ipsum. Aenean non felis malesuada, feugiat lectus a, tincidunt quam. Fusce nec quam egestas, vulputate est in, commodo nisi. Phasellus id nunc sit amet quam iaculis ornare eu id libero. Ut tempor nisi sed est pretium auctor. Donec in nunc turpis. Integer non tristique dolor. Curabitur a elit mollis urna finibus scelerisque sit amet vel erat. Nullam nec maximus erat. Duis ante mi, posuere ut lobortis nec, posuere eu ligula.

\subsection{Augmented Reality}

Having a 3D model of the historic Pellerhaus opens a whole new set of possible new ways to communicate the history. Example: "Timetraveler The Berlin Wall App" ( \url{https://www.youtube.com/watch?t=75&v=CY9f6UJZlmM})
