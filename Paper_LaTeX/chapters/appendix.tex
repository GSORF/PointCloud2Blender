\section{Software used}

\subsection{\LaTeX}
This paper was written in \LaTeX. On Windows, TeXstudio in conjunction with MikTeX (both portable versions) have been used for visual creation of the document. I decided to switch from the free version Adobe InDesign CS 2.0 to \LaTeX in favor of it being cross-platform and hoping to make it easier to publish the thesis online in the future. Since I have never worked with \LaTeX before, various tutorials \parencite{ytLaTeX,webLaTeX-Tutorial} on the internet have been a great help.

\subsection{Faro SCENE LT}
For preprocessing of the raw laser scanner point cloud.

\subsection{Blender 3D}
To cleanup the generated mesh, retopologize it and create the 3D animations of the Pellerhaus, Blender was used.

\subsection{Meshlab}
For converting binary test files to ASCII.

\subsection{Visual SfM}
For creating 3D models from images for free.

\subsection{Agisoft Photoscan Professional}
For creating 3D models from images with a 30-day test period. Mostly historic stereo pairs have been processed well.

\section{Programming frameworks and libraries}

\subsection{Qt 5.4}

Qt is an open source framework ...

\subsection{OpenGL}

The Open Graphics Library (OpenGL) is an abstraction layer for accessing graphics hardware on a high level. The current version supports the programmable function pipeline, where vertex and fragment shaders provide a rich set of ways to manipulate the final pixel on the output device, e.g. computer screens. ...