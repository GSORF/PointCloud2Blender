\section{Pellerhaus Architect Biography}
\label{appendix_pellerhaus_architects}

\subsection{German}

Körner, Hans-Michael and Jahn, Bruno (2012, p.2128) \parencite{bookBayerischeBiographische} wrote a detailed biography about the architects of the Pellerhaus Nürnberg:\\

\blockquote{
	
	\textbf{Wolff, Jakob d.Ä., Baumeister, Bildhauer, * um 1546 Bamberg, † 4.4.1612 Nürnberg} \\
	W. wurde 1596 Stadtbaumeister in Nürnberg, wo er mit seinem Sohn Jakob -> W. d.J. die Fleischbrücke errichtete. 1601-05 beteiligte er sich am Neubau der Feste Marienberg in Würzburg und am Umbau des Echtertors. Sein Hauptwerk ist das Pellerhaus in Nürnberg (1602-07), einer der vornehmsten Privatbauten der deutschen Renaissance (im Zweiten Weltkrieg zerstört; die Reste des Arkadenhofs wurden in den modernen Bau einbezogen).\\
	LITERATUR: Wilhelm Schwemmer: J.W. der Ältere und der Jüngere. In: Fränkische Lebensbilder. Bd. 3. Hrsg. v. Gerhard Pfeiffer. Würzburg 1969, S. 194-213.\\
	
	\textbf{Wolff, Jakob d.J., Baumeister, *1572 Bamberg, † 24.2.1620 Nürnberg} \\
	W. war Schüler seines Vaters Jakob -> W. d.Ä., erhielt 1605 in Nürnberg die Stelle eines Stadtwerkmeisters, hielt sich mit Erlaubnis des Rats u.a. in Bayreuth, Frauenaurach und Schwabach auf und begann, beeinflußt von der niederländischen und italienischen Renaissance, 1616 mit dem Neubau des Rathauses in Nürnberg, der 1622 von seinem Bruder Hans vollendet wurde. \\
	
	}

\subsection{English}

Körner, Hans-Michael and Jahn, Bruno (2012, p.2128) \parencite{bookBayerischeBiographische} translated from German by the author:

\blockquote{
	
	\textbf{Wolff, Jakob d.Ä., master builder, sculptor, *1546 Bamberg, † 4.4.1612 Nuremberg} \\
	Wolff became the city architect of Nuremberg in 1596, where he and his son W. d.J. built the Fleischbrücke. During 1601-05 he took part in the new build of the stronghold Marienberg in Würzburg and in the reconstruction of the Echtertor. His principal work is the Pellerhaus in Nuremberg (1602-07), one of the most noble private properties during the German Renaissance (destroyed in the Second World War; the remaining parts of the arcade court have been included in the modern building) [...] \\
	
	\textbf{Wolff, Jakob d.J., master builder, *1572 Bamberg, † 24.2.1620 Nuremberg} \\
	Wolff was the student of his father W. d.Ä., was given the job of a Stadtwerkmeister (Municipal Master of the Works) in 1605, had the permission from the council to stay in Bayreuth, Frauenaurach and Schwabach and started, influenced by the Dutch and Italian Renaissance, with the new build of the city hall in Nuremberg in 1616, which was finished by his brother Hans in 1622. \\
	
}





\section{Software used}

\subsection{\LaTeX}
This paper was written in \LaTeX. On Windows, TeXstudio in conjunction with MikTeX (both portable versions) have been used for visual creation of the document. I decided to switch from the free version Adobe InDesign CS 2.0 to \LaTeX in favor of it being cross-platform and hoping to make it easier to publish the thesis online in the future. Since I have never worked with \LaTeX before, various tutorials \parencite{ytLaTeX,webLaTeX-Tutorial} on the internet have been a great help.

\subsection{Faro SCENE LT}
For preprocessing of the raw laser scanner point cloud.

\subsection{Blender 3D}
To cleanup the generated mesh, retopologize it and create the 3D animations of the Pellerhaus, Blender was used.

\subsection{Meshlab}
For converting binary test files to ASCII.

\subsection{Visual SfM}
For creating 3D models from images for free.

\subsection{Agisoft Photoscan Professional}
For creating 3D models from images with a 30-day test period. Mostly historic stereo pairs have been processed well.

\section{Programming frameworks and libraries}

\subsection{Qt 5.4}

Qt is an open source framework ...

\subsection{OpenGL}

The Open Graphics Library (OpenGL) is an abstraction layer for accessing graphics hardware on a high level. The current version supports the programmable function pipeline, where vertex and fragment shaders provide a rich set of ways to manipulate the final pixel on the output device, e.g. computer screens. ...


