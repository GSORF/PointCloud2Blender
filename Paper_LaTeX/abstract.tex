\thispagestyle{plain}
	\begin{center}
		
		\LARGE
		\textbf{Abstract}
		
	\end{center}


In this work the interested reader will learn about my research on the 3D-model reconstruction of the historic Pellerhaus in Nuremberg, Germany, as it looked before its destruction during World War II. The title of this paper is "Visualization of laser scanner point clouds as 3D panoramas".

In the first chapter I will describe the background research that provided me with the necessary fundamentals to start the project.
The second chapter presents the development process of the software tools applied to achieve the goal of reconstructing historic 3D models from various data such as images and laser scans. To accomplish this, I decided to improve the open source software Blender.
Details on the production of a three-dimensional mesh from laser scans via LIDAR devices can be found in Chapter Three.
Chapter Four concludes the work and also presents future work. It contains the results, failures and successes of my research. Furthermore it discusses different possible ways to build upon the fundamental insights gained from this report.
Due to our modern open culture with several open software, hardware and movie projects - mainly inspired by the Blender Foundation - I also want to make my research available to the public. During the time I am writing my thesis I will therefore be publishing my progress online at http://bachelor.kalisz.co.

\pagebreak