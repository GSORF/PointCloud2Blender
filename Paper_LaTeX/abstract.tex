\thispagestyle{plain}
\begin{center}
	
	\LARGE
	\textbf{Abstract}
	
\end{center}
\vspace{100pt}

This study examines a novel approach to convert point clouds generated via laser scanning into textured 3D-meshes. The title of this paper is "Visualization of laser scanner point cloud as 3D panorama".
The approach is field-tested with a use case scenario where the interested reader will learn about our research on the 3D-model reconstruction of the historic Pellerhaus in Nuremberg, Germany, as it looked before its destruction during World War II.\\

Initially, the motivational force and background research that provided necessary fundamentals to start the project are described in the first chapter.
The second chapter presents the development process of the software tools applied to achieve the goal of reconstructing historic 3D models from various data such as images and laser scans. To accomplish this, a custom converter software has been written, which reads point cloud files and outputs the meshed and textured 3D-object file. The working title of this software is "PointCloud2Blender", \textit{PC2B} in short.
As a real world use case the creation of a photorealistic three-dimensional mesh from laser scans via LIDAR devices is described in detail in Chapter Three.
Chapter Four concludes the work and presents future work. It contains the results, failures and successes of this research. Furthermore it discusses different possible ways to build upon the fundamental insights gained from this report.\\

Due to our modern open culture with several open software, hardware and movie projects - mainly inspired by the Blender Foundation - this research is being made available to the public. During the time of the writing of this thesis the progress is therefore published online at http://bachelor.kalisz.co.
